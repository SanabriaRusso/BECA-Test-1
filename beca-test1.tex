% \documentclass[a4paper,journal]{IEEEtran}
\documentclass[conference]{IEEEtran}
%% INFOCOM 2013 addition:
\makeatletter
\def\ps@headings{%
\def\@oddhead{\mbox{}\scriptsize\rightmark \hfil \thepage}%
\def\@evenhead{\scriptsize\thepage \hfil \leftmark\mbox{}}%
\def\@oddfoot{}%
\def\@evenfoot{}}
\makeatother
\pagestyle{headings}
\usepackage{psfrag}

% \usepackage{auto-pst-pdf}

%for using the therefore symbol
\usepackage{amssymb}
%end


\usepackage[utf8]{inputenc}
\usepackage{graphicx}
\usepackage{float}
\usepackage{color, colortbl}
\usepackage{xcolor}
\usepackage{array}
\usepackage{multirow}
\usepackage{footnote}
\usepackage{cite}
%The below is used to add notes to tables without disrupting the IEEEtran format
\usepackage{threeparttable}
%Multiple figures in a row
\usepackage{caption}
\usepackage{subcaption}

% Disable below if wanting to comply exclusively to conference mode of IEEEtran
% \IEEEoverridecommandlockouts

\makesavenoteenv{tabular}

%Ignores \vbox errors below the level of 10000
% \vbadness=10000
%DIF PREAMBLE EXTENSION ADDED BY LATEXDIFF
%DIF UNDERLINE PREAMBLE %DIF PREAMBLE
\RequirePackage[normalem]{ulem} %DIF PREAMBLE
\RequirePackage{color}\definecolor{RED}{rgb}{1,0,0}\definecolor{BLUE}{rgb}{0,0,1} %DIF PREAMBLE
\providecommand{\DIFadd}[1]{{\protect\color{red}\sout{#1}}} %DIF PREAMBLE
\providecommand{\DIFdel}[1]{{\protect\color{blue}\uwave{#1}}}                      %DIF PREAMBLE
%DIF SAFE PREAMBLE %DIF PREAMBLE
\providecommand{\DIFaddbegin}{} %DIF PREAMBLE
\providecommand{\DIFaddend}{} %DIF PREAMBLE
\providecommand{\DIFdelbegin}{} %DIF PREAMBLE
\providecommand{\DIFdelend}{} %DIF PREAMBLE
%DIF FLOATSAFE PREAMBLE %DIF PREAMBLE
\providecommand{\DIFaddFL}[1]{\DIFadd{#1}} %DIF PREAMBLE
\providecommand{\DIFdelFL}[1]{\DIFdel{#1}} %DIF PREAMBLE
\providecommand{\DIFaddbeginFL}{} %DIF PREAMBLE
\providecommand{\DIFaddendFL}{} %DIF PREAMBLE
\providecommand{\DIFdelbeginFL}{} %DIF PREAMBLE
\providecommand{\DIFdelendFL}{} %DIF PREAMBLE
%DIF END PREAMBLE EXTENSION ADDED BY LATEXDIFF

\begin{document}
%opening
 \title{Prototyping Collision-Free MAC Protocols in Real Hardware}


%A more simple output, useful when involving people from different affiliations
  %\author{
    %  \IEEEauthorblockN{Luis Sanabria-Russo\IEEEauthorrefmark{0}, Jaume Barcelo\IEEEauthorrefmark{0}, Boris Bellalta\IEEEauthorrefmark{0}}\\
      %\IEEEauthorblockA{\IEEEauthorrefmark{0}Universitat Pompeu Fabra, Barcelona, Spain
      %\\\{luis.sanabria, jaume.barcelo, boris.bellalta\}@upf.edu}
  %}

\author{Luis Sanabria-Russo \\
		NeTS Research Group at\\
		Universitat Pompeu Fabra, Barcelona, Spain\\
		\texttt{Luis.Sanabria@upf.edu}}

%This is the style of three columns, as indicated in IEEEtran
% \author{\IEEEauthorblockN{Luis Sanabria-Russo}
%  \IEEEauthorblockA{Department of Information\\
%  and Communications Technologies\\
%  Universitat Pompeu Fabra\\
%  Barcelona, Spain\\
%  Email: luis.sanabria@upf.edu}
%  \and
%  \IEEEauthorblockN{Jaume Barcelo}
%  \IEEEauthorblockA{Department of Information\\
%  and Communications Technologies\\
%  Universitat Pompeu Fabra\\
%  Barcelona, Spain\\
%  Email: cristina.cano@upf.edu}
%  \and
%  \IEEEauthorblockN{Boris Bellalta}
%  \IEEEauthorblockA{Department of Information\\
%  and Communications Technologies\\
%  Universitat Pompeu Fabra\\
%  Barcelona, Spain\\
%  Email: boris.bellalta@upf.edu}}


\maketitle

\begin{abstract}

\boldmath CSMA/CA is the current Medium Access Control (MAC) standard for orchestrating transmissions in WLANs. It has successfully performed for many years, making WiFi an ubiquitous wireless technology built with cheap hardware and very simple code. In the past five years many breakthroughs in the physical layer (PHY) caused a dramatic increase in throughput, allowing transmission speeds of over 300Mbps. Nevertheless, CSMA/CA dynamics require long headers, acknowledgements and contention periods to successfully transmit a single frame of user-generated data; reducing the benefits provided by a very fast PHY. Many amends have been proposed to leverage the ``MAC-bottleneck'' and sequentially incorporated into the standard. This report aims at providing an introduction to today's open tools that will allow any researcher to test MAC protocols in real hardware.

\end{abstract}

\begin{IEEEkeywords}
OpenFWWF, WMP, MAC, Collision-free, CSMA/ECA.
\end{IEEEkeywords}

\section*{A short warning} \label{warning}
Prior the introduction, it is appropriate to filter interests. This report assumes a bit of background on WiFi technology and terminology, nevertheless many of the references are detailed at the end of the document.

Procedures described here must be done at your own risk. Wireless cards (as mentioned in some of the references) might get permanently damaged. Nevertheless, all the events and workarounds that were necessary to achieve the final test of CSMA/ECA will be dutifully detailed.

Now, keep on reading :).

\section{Introduction}\label{introduction}
A device firmware is the one managing memory and code to make the device perform as intended. As for most devices, including wireless cards, the firmware is custom made, unique for each architecture.

Even-though the IEEE 802.11 set of WLAN standards define the procedures to guarantee effective communication among hosts, the implementation part is the task of manufacturers. So, it is clear why firmware is closely related to the underlying hardware and how many different hardware/firmware combinations may perform the same function set.

To protect manufacturer's commercial interests, firmwares are not usually allowed to be modified until the end of the product's life-cycle. Nevertheless, current efforts both from the industry and the open source community (as in the case of MadWiFi~\cite{madWiFi} and OpenFWWF~\cite{OpenFWWF}), had led to interesting opportunities for the research community.

This document will guide you through the process of acquiring and pushing the open sourced OpenFWWF firmware to an specific model of Wireless Network Interface Controller (WNIC or wireless cards). Most of the procedures described here can be found on the web, other recommendations are relayed from one of the authors of OpenFWWF.

\section{What will you need?}\label{hardware}
As mentioned in~\cite{OpenFWWF}, it is recommended to use a Linux distribution with kernel version 2.6.27-rc5. Feel free to find any source of old Linux versions. Nevertheless, it has proved to be safe for us to use Ubuntu 8.10~\cite{ubuntu8}.

The OpenFWWF is used in combination with the \texttt{b43} Linux  wireless card driver, which in turn is supported by a limited set of Broadcom cards~\cite{b43-info}. For this reason, you will need such card model to continue. We have successfully tested our implementation with a Broadcom BCM4318 card (see Figure~\ref{fig:miniPCI}), which is available at~\cite{bcm4318} and connected to a PCI slot with a miniPCI to PCI adapter (see Figure~\ref{fig:PCI}), like the one available at~\cite{PCIAdapter}.

It is also important to have a wired Internet connection to continue further, given that after the Ubuntu installation your are required to fetch the firmware, compiler and prerequisites.

\begin{figure*}[t]
\centering
\begin{subfigure}{.5\textwidth}
  \centering
  \includegraphics[width=\linewidth]{figures/miniPCI.eps}
  \caption{}
  \label{fig:miniPCI}
\end{subfigure}%
\begin{subfigure}{.5\textwidth}
  \centering
  \includegraphics[width=0.9\linewidth]{figures/cardPCI.eps}
  \caption{}
  \label{fig:PCI}
\end{subfigure}
\caption{\ref{fig:miniPCI}) Broadcom BCM418 miniPCI. \ref{fig:PCI}) Card correctly placed into the PCI adapter.}
\label{installedCard}
\end{figure*}

\section{Setting up the environment}
After the installation of the Ubuntu 8.10, restrain from installing updates. It is better to complete the procedure as detailed here and then go on your own.

Because Ubuntu 8.10 is very old, you may need to update the \texttt{/etc/apt/sources.list} file to include a repository where you can download and install the prerequisites. Open a Terminal window and type: \texttt{sudo gedit /etc/apt/sources.list}. This command will cause the Gedit text editor to open the specified file with root credentials (you may be prompted for the root password). Go to the end of the file and add the following line: \texttt{deb http://ubuntu.mirror.cambrium.nl/ubuntu/ lucid main}. Save and close the file.

Now you need to update the sources list by issuing the following command in the Terminal (you must be connected to the Internet): \texttt{sudo apt-get update} (you may be prompted for the root password). Depending on your Internet connection, this may take a while.

After the update completes, you are now able to download the prerequisites. It is required to install the \texttt{git-core} package to fetch the code from remote sources, \texttt{g++} to compile C++ code, \texttt{bison} which is a parser generator and \texttt{flex} (Fast Lexical Analyzer). Issue the following command in a terminal window: \texttt{sudo apt-get install git-core bison flex g++} (you may be prompted for the root password). This takes some time to complete.

After all the above, your computer is ready for the card and firmware installation.

\subsection{Installing the required hardware}

If you are using the card referred to in Section~\ref{hardware}, it is necessary to plug it into a miniPCI to PCI adapter. This is a very delicate task, and you should avoid forcing the parts into place. Please refer to Figure~\ref{installedCard} to see the finished result.

\subsection{Setting OpenFWWF to work}
There are various ways of completing this part of the procedure. Nevertheless, the one provided by~\cite{gnewsense} is both simple and it works.

Summarizing (links refer to the actual commands for layout limitations):
\begin{enumerate}
	\item Download the latest version of the firmware source code by issuing the command contained in~\texttt{http://pastebin.com/VwdKEhZ1} into a Terminal window. \\ This will create a directory named  openfwwf-5.2.tar.gz containing the OpenFWWF firmware. Unpack it issuing the following command:~\texttt{tar -zxvf openfwwf-5.2.tar.gz}.
	\item Download the assembly language compiler (b43-tools~\cite{b43-tools}) from its git repository issuing the command found in \texttt{http://pastebin.com/8RC6c01g}
	\item Once \texttt{b43-tools} is downloaded, issue \texttt{cd b43-tools/assembler} to get into the assembler directory. Then build it typing the \texttt{make} command.\\ This step will create two files, namely: \texttt{b43-asm} and \texttt{b43-asm.bin}.
	\item Copy \texttt{b43-asm} and \texttt{b43-asm.bin} into the openfwwf-5.2 directory by typing \texttt{cp b43-tools/assembler/b43-asm* openfwwf-5.2/}
	\item You need to modify the Makefile (openfwwf-5.2/Makefile) replacing \texttt{BCMASM = b43-asm} by \texttt{BCMASM = ./b43-asm}.
	\item Now just build the firmware and install it by typing \texttt{make} and then \texttt{sudo make install}.\label{make}
	\item As recommended in~\cite{OpenFWWF}, edit /etc/modprobe.d/arch/i386 file issuing the following command to open it~\texttt{sudo gedit etc/modprobe.d/arch/i386} and then add the following line to the end of the file: \texttt{options b43 qos=0 nohwcrypt=1}. Save and close the file.
	\item Restart your computer.
\end{enumerate}

You can browse and edit the firmware by modifying the openfwwf-5.2/ucode5.asm file. To test the OpenFWWF, build and install the firmware (as in Step~\ref{make} above) and restart your computer.

Now you should be able to see the wireless interface at the top right corner of Ubuntu's menu bar when you log back in into your user account.

\section{Prototyping and testing a new MAC protocol}
Carrier Sense Multiple Access with Enhanced Collision Avoidance (CSMA/ECA) was first introduced by Barcelo et. al. in~\cite{barcelo2008lba} and later adapted to support a greater number of contenders~\cite{research2standards}. 

To better describe CSMA/ECA it is better to review CSMA/CA:

\begin{enumerate}
	\item When nodes have something to transmit, they generate a random backoff $B\in[0,CW(k)]$. Where $CW(k)=2^{k}CW_{min}$ is the Contention Window at backoff stage $k\in[0,m]$ and $CW_{min}$ the minimum contention window.
	\item Each passing empty slot decrements $B$ in one. When the backoff expires ($B=0$), the node will attempt transmission.
	\item If an ACKowledgement (ACK) from the receiver is received: the backoff stage is reset ($k=0$). And if there are other packets in the MAC queue, the backoff process is restarted.\label{reset}
	\item If no ACK is received, a collision is assumed. The backoff stage is incremented in one ($k$+=1) and the backoff process is restarted.
\end{enumerate}

Based on the description of CSMA/CA presented above, CSMA/ECA just picks a deterministic backoff $B_{d}=CW_{min}/2$ after successful transmissions; collisions are handled as in CSMA/CA.

A short example of what is proposed in~\cite{barcelo2008lba} is shown in Figure~\ref{Basic-ECA}.

%Instead of picking a random backoff $B\in[0,2^{k}CW_{min}]$ (where $k\in[0,m]$ is the backoff stage incrementing with each collision up until $m$ and reset to zero after a succesfful transmission; $CW_{min}$ being the minimum contention window) as CSMA/CA does, nodes running CSMA/ECA pick a deterministic backoff $B_{d}=CW_{min}/2$ after successful transmissions. This way successful contenders will not collide among each other in future cycles. A short example of what is proposed in~\cite{barcelo2008lba} is shown in Figure~\ref{Basic-ECA}.

\begin{figure}[htbp]
\centering
  \includegraphics[width=\linewidth]{figures/basicECA.eps}
  \caption{CSMA/ECA example in saturation}
  \label{Basic-ECA}
\end{figure}

In Figure~\ref{Basic-ECA}, each station (STA) generates a random backoff at startup. The red outline indicates that at least two of the generated backoff values are the same and stations will consequently collide. On the other hand, successful stations will generate a deterministic backoff ($7$ in the case of example Figure~\ref{Basic-ECA}) and effectively avoid collisions with other successful stations in future cycles, achieving a collision-free state.

A collision-free WLAN carries with it many benefits in terms of throughput when compared with CSMA/CA. Furthermore, the simplicity of CSMA/ECA makes it very easy to implement in OpenFWWF, as will be described in the following.

\subsection{Modifying the backoff mechanism}
To replicate CSMA/ECA behaviour, each station must pick the same deterministic backoff after successful transmissions. The backoff parameters are set in the \texttt{set\_backoff\_time} function of the \emph{ucode.asm} file (near the end).

To make sure CSMA/ECA stations pick a deterministic backoff after successful transmission, we first check that the current contention window is the minimum contention window (see Item~\ref{reset} in the CSMA/CA description). Then, a deterministic backoff is assigned. The following shows the replaced code as well as the code lines that would convert CSMA/CA into CSMA/ECA in OpenFWWF.

\begin{itemize}
	\item Random backoff generation: {\bf and} SPR\_TSF\_Random, CUR\_CONTENTION\_WIN, GP\_REG5; is replaced by: {\bf je} CUR\_CONTENTION\_WIN, DEFAULT\_MIN\_CW, deterministic\_backoff; \\
	This replacement just checks if the current contention window is equal to the minimum contention window ({\bf if}(CUR\_CONTENTION\_WIN == DEFAUL\_MIN\_CW)). If so, then we can assume that the station either successfully transmitted a packet or just got a packet into a previously empty MAC queue.
	\item If the above result is positive, the flow is redirected to the deterministic\_backoff section of the \texttt{set\_backoff\_time} function, which contains the following instruction: {\bf or} 0x100, 0x000, GP\_REG5;. Assigning 0x100 to the GP\_REG5 register and successfully changing the backoff value.
\end{itemize}

\subsection{Testing CSMA/ECA}
We built a simple testing scenario to check whether the modifications performed matched the expected CSMA/ECA behaviour. This was composed of two Ubuntu 8.10 PCs with Broadcom BCM4318 cards running OpenFWWF firmware as WLAN STAtions (STA) connected to an Access Point (AP). To make performance tests, Iperf~\cite{tirumala2005iperf} tool generates UDP streams at 65 Mbps  from both STAs to a Server wired to the access point using Ethernet. At the Server, Wireshark~\cite{combs2007wireshark} captures all packets from the STAs. Figure~\ref{setup} provides an overview of the testing scenario.

\begin{figure}[htbp]
\centering
  \includegraphics[width=0.7\linewidth]{figures/test.eps}
  \caption{Test setup}
  \label{setup}
\end{figure}

Two simple test were performed, the first tries to reveal evidence of the deterministic backoff counter, while the second aims at looking at the offered throughput. 

Figures~\ref{ca-backoff}~and~\ref{eca-backoff} show a random set of a hundred server-received packets from STAs running CSMA/CA and CSMA/ECA, respectively.

\begin{figure}[htbp]
\centering
  \includegraphics[width=0.7\linewidth,angle=-90]{figures/csma-2h-small.eps}
  \caption{CSMA/CA transmission turns between node 66 and 65}
  \label{ca-backoff}
\end{figure}

\begin{figure}[htbp]
\centering
  \includegraphics[width=0.7\linewidth,angle=-90]{figures/BECA8-2h-small.eps}
  \caption{CSMA/ECA transmission turns between node 66 and 65}
  \label{eca-backoff}
\end{figure}

\subsection{Results}

\subsubsection*{Backoff mechanism}
As it is expected, CSMA/CA's randomised backoff mechanism is easy to appreciate in Figure~\ref{ca-backoff}, where the "Transmitters" line shows the transmitter of given packet. Whereas in CSMA/ECA (Figure~\ref{eca-backoff}), transmitters seem to alternate transmissions.

\subsubsection*{Throughput}
From the "Difference of arrivals" curve, we can appreciate that with CSMA/CA the average time between arrived packets is greater than with CSMA/ECA. This means that CSMA/ECA stations are able to access the channel more often than those running CSMA/CA, resulting in an increased throughput. Figure~\ref{throughput} shows the average throughput achieved by each station while attempting to transmit generic UDP packets at 65 Mbps to the Server.

\begin{figure}[htbp]
\centering
  \includegraphics[width=0.7\linewidth,angle=-90]{figures/throughput-bar-small.eps}
  \caption{Throughput}
  \label{throughput}
\end{figure}

\section{Conclusion}
OpenFWWF is a open alternative for researchers wanting to test new MAC protocols in real hardware. In this document the process of obtaining, installing and testing a novel MAC protocol is described alongside with encouraging results.

Carrier Sense Multiple Access with Enhanced Collision Avoidance (CSMA/ECA) attempts to build a collision-free WLAN environment, which has significant benefits in terms of throughput. Results from these tests show evidence of an effective modification of CSMA/CA's backoff mechanism alongside with the throughput increase of CSMA/ECA.

This is the first real hardware implementation of CSMA/ECA. Further work might involve the design of more accurate testing scenarios, collision detection and the extension of CSMA/ECA to work with more contenders~\cite{research2standards}.

\section{Acknowledgement}
The author would like to extend appreciation to Francesco Gringoli for his insight and counsel, as wellgit  to Germán Corrales Madueño for his ever-pushing and contagious motivation.

\bibliographystyle{Classes/IEEEtran}
\bibliography{IEEEabrv,ref}

\end{document}